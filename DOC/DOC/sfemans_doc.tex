\documentclass[a4paper]{book}

\usepackage{a4wide}
\usepackage{mydef,boldfonts}
\usepackage{amssymb}
\usepackage{amsfonts}
\usepackage{amsmath,color}
%\usepackage{pstricks}
\usepackage{graphics,comment,fancybox}
\usepackage{verbatim}
\usepackage{hyperref}
\usepackage{multirow,multicol}
\newcounter{number_tests_sfemans}
\setcounter{number_tests_sfemans}{17}
\newcounter{tests_sfemans}
\setcounter{tests_sfemans}{0}
%!!!!!!!!!!!!!!!!!!!!!!!!!!!!!!!!!!!!!!!!!!!!!!!!!!!!!!!!!!!!!!!!!!!!!!!!!!!!!
% !!! DCQ Comment:  counters does not work proeorly with the \chapter command, 
%                   use  testChapter instead.
\newcommand{\testChapter}[1]{
\stepcounter{tests_sfemans}
\chapter{ Test  \arabic{tests_sfemans}:  #1}
}
%!!!!!!!!!!!!!!!!!!!!!!!!!!!!!!!!!!!!!!!!!!!!!!!!!!!!!!!!!!!!!!!!!!!!!!!!!!!!!


\setcounter{tocdepth}{2}
\def\data_dir{../MHD_DATA_TEST_CONV_PETSC}

\title{SFEMaNS user guide}

\author{FL, JLG, CN, DCQ ...}
\date{\today}

\begin{document}
\newcommand{\vertical}[1]{\rotatebox{90}{\makebox{#1}}}

\maketitle
\tableofcontents
\section{General presentation}
\subsection{Introduction}
\subsection{System of equations}
\subsection{General features of SFEMaNS}

\chapter{Installing SFEMaNS}
\subsection{How to obtain SFEMaNS ?}
\subsection{External tools}
\subsection{Installing PETSc with MUMPS}
\subsection{ARPACK library}
The installation of the libraries ARPACK and PARPACK is done following these steps:
\begin{itemize}
\item Download \verb?arpack96.tar.gz ... ppatch.tar.gz? from the url ...
\item Uncompress the archives in the following order,
\item In the directory \verb?ARPACK?, replace the file \verb?ARmake.inc? by one of the templates in the subdirectory \verb?ARMAKES?, according to your environment,
\item Edit the file \verb?ARmake.inc? to make it compatible with your environment: in particular, the path for the top-level directory of ARPACK, make sure the Fortran compilers (\verb?FC? and \verb?PFC?) are the right ones, and check the path for the command \verb?make?,
\item Build the libraries, using \verb?make lib? and \verb?make plib?.
\end{itemize}
\section{Mesh generator}
\section{Adaptation to user's environment}

x\chapter{Getting started}
\section{First run}
Begin with SFEMaNS by creating a directory (e.g. \verb?MY_APPLICATION?) in which you want to create the executable file. Then you need to :
\begin{itemize}
\item Copy the content of \verb?$(HOME_SFEMaNS)/TEMPLATE? into \verb?MY_APPLICATION?,
\item Edit the file \verb?my_make? to make it compatible with your environment, by specifying the right path to the top level directory of SFEMaNS,
\item Edit the file \verb?make.inc? to make it compatible with your environment: in particular, if you have installed ARPACK and PARPACK on your own, you need to use
\begin{verbatim}
PA_LIB    =  $(HOME_ARPACK)/name_of_your_P_arpack_lib $(HOME_ARPACK)/name_of_your_arpack_lib
\end{verbatim}
in this specific order,
\item Create the file \verb?makefile? using the command: \verb?./my_make?,
\item Create the executable file \verb?a.exe? using the command: \verb?make a.exe?.
\end{itemize}
\section{Type of problem}
SFEMaNS can solve three different types of problems, listed below.
\subsection{Type \texttt{nst}}
\subsection{Type \texttt{mxw}}
\subsection{Type \texttt{mhd}}
\section{Numerical domain}
\section{\texttt{data} file}
The file \texttt{data} contains all the data needed for the computation. The user has to specify the geometry of the domain, the list of conductive and insulating parts, as well as general information for the parallel runs. The \texttt{data} file is divided in blocks : one block is mandatory, some others are needed depending on the type of the problem you want to solve, and a couple of blocks are optional. Table~\ref{tab:data_blocks} summarizes the blocks needed for a SFEMaNS run.
\subsection{Required informations}
\subsection{Type related blocks}
\subsection{Optional blocks}
\begin{table}[h]
\begin{tabular}{|l|cccccccc|}
\hline
 &\multicolumn{3}{c|}{Basic use}&\multicolumn{5}{c|}{Advanced computations} \\ \hline
Blocks & \vertical{nst} & \vertical{mxw} &\multicolumn{1}{c|}{\vertical{mhd}} & \vertical{nst with temperature} & \vertical{mxw without $\phi$} & \vertical{mhd with temperature} & \vertical{mhd without $\phi$} & \vertical{Arpack on $\bH$} \\ \hline
\texttt{GENERAL DATA}                  & R & R & R & R & R & R & R & R\\ \hline
\texttt{Mesh-NAVIER-STOKES}            & R & O & R & R & O & R & R & X\\ \hline
\texttt{BCs-NAVIER-STOKES}             & R & X & R & R & X & R & R & X\\ \hline
\texttt{Dynamics-NAVIER-STOKES}        & R & X & R & R & X & R & R & X\\ \hline
\texttt{LES-NAVIER-STOKES}             & R & X & R & R & X & R & R & X\\ \hline
\texttt{Solver-velocity-NAVIER-STOKES} & O & X & O & O & X & O & O & X\\ \hline
\texttt{Solver-pressure-NAVIER-STOKES} & O & X & O & O & X & O & O & X\\ \hline
\texttt{Solver-mass-NAVIER-STOKES}     & O & X & O & O & X & O & O & X\\ \hline
\texttt{Verbose (diagnostics)}         & O & O & O & O & O & O & O & O\\ \hline
\texttt{Solver-MAXWELL}                & X & O & O & O & O & O & O & O\\ \hline
\texttt{H-MAXWELL}                     & X & R & R & X & R & R & R & R\\ \hline
\texttt{Phi-MAXWELL}                   & X & R & R & X & X & O & X & O\\ \hline
\texttt{Verbose-MAXWELL}               & X & O & O & X & O & O & O & C\\ \hline
\texttt{Phase}                         & O & X & X & R & X & R & O & X\\ \hline
\texttt{Solver-Phase}                  & O & X & O & O & X & O & O & X\\ \hline
\texttt{Post-processing}               & O & O & O & O & O & O & O & X\\ \hline
\texttt{Periodicity}                   & O & O & O & O & O & O & O & O\\ \hline
\texttt{ARPACK}                        & X & X & X & X & X & X & X & R\\ \hline
\texttt{Visualization}                 & X & X & X & X & X & X & X & O\\ \hline
\texttt{BLOCK}                         & C & C & C & C & C & C & C & C\\ \hline
\end{tabular}
\caption{Summary of the blocks in the \texttt{data} file (R=Required, O=Optional, X=Useless)}
\label{tab:data_blocks}
\end{table}

\section{Boundary conditions (\texttt{condlim.f90} file)}

\section{The main program (\texttt{main.f90} file)}

\chapter{Tools}
\section{Backup tools}
\section{Visualization tools}
\section{Variables in SFEMaNS}
\section{Custom variables}
Using \texttt{read\_user\_data.f90}, SFEMaNS allows the use of custom variables. All the custom variables have to be declared in the type \texttt{user\_data\_type} and can be used in the \texttt{main.f90} file. If needed, the user can read personal data from a file, either by appending the \texttt{data} file, or by creating another one. User's data file (e.g. \texttt{my\_own\_data}) has to be read in the \texttt{main.f90} with the following
\begin{verbatim}
CALL read_user_data('my_own_data')
\end{verbatim}
After this call, all the variables declared in the type \texttt{user\_data\_type} can be used with the prefix \verb?my_data%?.

Use the template in \texttt{read\_user\_data.f90} to add any number of variables.
\chapter{Tests in SFEMaNS}
The command
\begin{verbatim}
./debug_SFEMaNS
\end{verbatim}
is used to run \arabic{number_tests_sfemans} different
tests. Informations about these cases are listed below.

\testChapter{nst}
%\chapter{Test \testNumber: nst.}
\section{Numerical domain and equations to solve}
\begin{align*}
\partial_t\bu+\left(\ROT\bu\right)\CROSS\bu - \frac{1}{\Re}\LAP \bu +\GRAD p &=\bef,
\\ \DIV \bu &= 0, \\
\bu_{|\Gamma} &= \bv , \\
\bu_{|t=0} &= \bu_0.
\end{align*}
The data are $\bef$, $\bv$ and $\bu_0$. We use $\Re=1$.
\section{Analytical solution}
\begin{align*}
u_r(r,\theta,z,t) &= \left(\left(r^2z^3-3r^3z^2\right)\cos(\theta) -
\left(r^2z^3+3r^3z^2\right)\sin(\theta)\right)\cos(t),
\\ u_{\theta}(r,\theta,z,t) &=
3\left(r^3z^2-r^2z^3\right)\left(\cos(\theta)+\sin(\theta)\right)\cos(t),
\\ u_z(r,\theta,z,t) &=
\left(3r^2z^3\cos(\theta)+5r^2z^3\sin(\theta)\right),
\\ p(r,\theta,z,t) &= rz\left(\cos(\theta)+\sin(\theta)\right)\sin(t).
\end{align*}
\section{Data file}
\verbatiminput{./\data_dir/debug_data_1}
%%%%END OF TEST 1
%
%
%
\testChapter{nst + perio}
%\chapter{Test \testNumber: nst + perio.}
\section{Numerical domain and equations to solve}
\begin{align*}
\partial_t\bu + \left(\ROT\bu\right)\CROSS\bu - \frac{1}{\Re}\LAP \bu +\GRAD p &=\bef,
\\ \DIV \bu &= 0, \\
\bu_{|\Gamma} &= \bv , \\
\bu_{|t=0} &= \bu_0, \\
\bu\textnormal{ periodic in the }z\textnormal{-direction}.&
\end{align*}
\section{Analytical solution}
\begin{align*}
u_r(r,\theta,z,t) &= -r^2\left(1-2\pi r\sin(2\pi z)\right)\sin(\theta)\cos(t), \\
u_{\theta}(r,\theta,z,t) &= -3r^2\cos(\theta)\cos(t), \\
u_z(r,\theta,z,t) &= r^2\left(4\cos(2\pi z)+1\right)\sin(\theta)\cos(t), \\
p(r,\theta,z,t) &= r^2\cos(2\pi z)\cos(\theta)\cos(t).
\end{align*}
\section{Data file}
\verbatiminput{\data_dir/debug_data_2}
%%%%END OF TEST 2
%
%
%
\testChapter{mxw (with vacuum).}
%\chapter{Test \testNumber: mxw (with vacuum).}
\section{Numerical domain and equations to solve}
\begin{align*}
\partial_t\left(\mu\bH\right)+\frac{1}{\Rm}\ROT\left(\frac{1}{\sigma}\ROT\bH\right)
- \ROT\left(\bu\CROSS\mu\bH\right) &=
\frac{1}{\Rm}\ROT\left(\frac{1}{\sigma}\bj\right), \\ \DIV(\mu\bH) &=
0, \\ +BC+IC. & +\phi
\end{align*}
\section{Analytical solution}
\begin{align*}
H_r(r,\theta,z,t) &= \sum_{m=1}^{3} \frac{1}{m^3}\left(\alpha_m
zr^{m-1} m\cos(m\theta) + \beta_m zr^{m-1}
m\sin(m\theta)\right)\cos(t), \\ H_{\theta}(r,\theta,z,t) &=
\sum_{m=1}^{3} \frac{1}{m^3}\left(\beta_m zr^{m-1} m\cos(m\theta) -
\alpha_m zr^{m-1} m\sin(m\theta)\right)\cos(t),\\ H_z(r,\theta,z,t) &=
\sum_{m=1}^{3} \frac{1}{m^3}\left(\alpha_m r^m \cos(m\theta) + \beta_m
r^m \sin(m\theta) \right)\cos(t),\\ \phi(r,\theta,z,t) &=
\sum_{m=1}^{3} \frac{1}{m^3}\left(\alpha_m zr^{m}\cos(m\theta) +
\beta_m zr^m\sin(m\theta)\right)\cos(t).
\end{align*}
\section{Data file}
\verbatiminput{\data_dir/debug_data_3}
%%%%END OF TEST 3
%
%
%
\testChapter{mxw + perio (with vacuum).}
%\chapter{Test \testNumber: mxw + perio (with vacuum).}
\section{Numerical domain and equations to solve}
\begin{align*}
\partial_t\left(\mu\bH\right)+\frac{1}{\Rm}\ROT\left(\frac{1}{\sigma}\ROT\bH\right) -
\ROT\left(\bu\CROSS\mu\bH\right) &=
\frac{1}{\Rm}\ROT\left(\frac{1}{\sigma}\bj\right), 
\\ \DIV(\mu\bH) &= 0, \\
+BC+IC+perio & +\phi
\end{align*}
\section{Analytical solution}
\begin{align*}
H_r(r,\theta,z,t) &= \cos(t)\cos(\theta)\cos(2\pi z)\left(\frac{r}{r_0^2}-2\pi\left(\frac{r}{r_0}\right)^2\left(A + B\frac{r}{r_0}\right)\right), \\
H_{\theta}(r,\theta,z,t) &= \cos(t)\sin(\theta)\cos(2\pi z)\left(2\pi\left(\frac{r}{r_0}\right)^2 C - 2\frac{r}{r_0^2}\right) ,\\
H_z(r,\theta,z,t) &= \cos(t)\cos(\theta)\sin(2\pi z)\frac{r}{r_0^2}\left(3A+4B\frac{r}{r_0}-C\right),\\
\phi(r,\theta,z,t) &= \cos(t)\cos(\theta)\cos(2\pi z) K_1(2\pi r),
\end{align*}
$K_1$: Bessel function.  $A,B,C$ constants.
\section{Data file}
\verbatiminput{\data_dir/debug_data_4}
%%%%END OF TEST 4
%
%
%
\testChapter{mxw (with vacuum).}
%\chapter{Test \testNumber: mxw (with vacuum).}
\section{Numerical domain and equations to solve}
\begin{align*}
\partial_t\left(\mu\bH\right)+\frac{1}{\Rm}\ROT\left(\frac{1}{\sigma}\ROT\bH\right) -
\ROT\left(\bu\CROSS\mu\bH\right) &=
\frac{1}{\Rm}\ROT\left(\frac{1}{\sigma}\bj\right), 
\\ \DIV(\mu\bH) &= 0, \\
+BC+IC. & +\phi
\end{align*}
\section{Analytical solution}
\begin{align*}
H_r(r,\theta,z,t) &= \sum_{m=1}^{3} \frac{1}{m^3}\left(\alpha_m zr^{m-1} m\cos(m\theta) + \beta_m zr^{m-1} m\sin(m\theta)\right)\cos(t), \\
H_{\theta}(r,\theta,z,t) &=  \sum_{m=1}^{3} \frac{1}{m^3}\left(\beta_m zr^{m-1} m\cos(m\theta) - \alpha_m zr^{m-1} m\sin(m\theta)\right)\cos(t),\\
H_z(r,\theta,z,t) &=  \sum_{m=1}^{3} \frac{1}{m^3}\left(\alpha_m r^m \cos(m\theta) + \beta_m r^m \sin(m\theta) \right)\cos(t),\\
\phi(r,\theta,z,t) &= \sum_{m=1}^{3} \frac{1}{m^3}\left(\alpha_m zr^{m}\cos(m\theta) + \beta_m zr^m\sin(m\theta)\right)\cos(t).
\end{align*}
\section{Data file}
\verbatiminput{\data_dir/debug_data_5}
%%%%END OF TEST 5
%
%
%
\testChapter{mxw (with vacuum).}
%\chapter{Test \testNumber: mxw (with vacuum).}
\section{Numerical domain and equations to solve}
\begin{align*}
\partial_t\left(\mu\bH\right)+\frac{1}{\Rm}\ROT\left(\frac{1}{\sigma}\ROT\bH\right) -
\ROT\left(\bu\CROSS\mu\bH\right) &=
\frac{1}{\Rm}\ROT\left(\frac{1}{\sigma}\bj\right), 
\\ \DIV(\mu\bH) &= 0, \\
+BC+IC. & +\phi
\end{align*}
\section{Analytical solution}
\section{Data file}
\verbatiminput{\data_dir/debug_data_6}
%%%%END OF TEST 6
%
%
%
\testChapter{mxw (with vacuum).}
%\chapter{Test \testNumber: mxw (with vacuum).}
\section{Numerical domain and equations to solve}
\begin{align*}
\partial_t\left(\mu\bH\right)+\frac{1}{\Rm}\ROT\left(\frac{1}{\sigma}\ROT\bH\right) -
\ROT\left(\bu\CROSS\mu\bH\right) &=
\frac{1}{\Rm}\ROT\left(\frac{1}{\sigma}\bj\right), 
\\ \DIV(\mu\bH) &= 0, \\
+BC+IC. & +\phi
\end{align*}
\section{Analytical solution}
\section{Data file}
\verbatiminput{\data_dir/debug_data_7}
%%%%END OF TEST 7
%
%
%
\testChapter{nst + phase.}
%\chapter{Test \testNumber: nst + phase.}
\section{Numerical domain and equations to solve}
\section{Analytical solution}
\section{Data file}
\verbatiminput{\data_dir/debug_data_8}
%%%%END OF TEST 8
%
%
%
\testChapter{nst + phase + perio.}
%\chapter{Test \testNumber: nst + phase + perio.}
\section{Numerical domain and equations to solve}
\section{Analytical solution}
\section{Data file}
\verbatiminput{\data_dir/debug_data_9}
%%%%END OF TEST 9
%
%
%
\testChapter{mxw (without vacuum).}
%\chapter{Test \testNumber: mxw (without vacuum).}
\section{Numerical domain and equations to solve}
\begin{align*}
\partial_t\left(\mu\bH\right)+\frac{1}{\Rm}\ROT\left(\frac{1}{\sigma}\ROT\bH\right) -
\ROT\left(\bu\CROSS\mu\bH\right) &=
\frac{1}{\Rm}\ROT\left(\frac{1}{\sigma}\bj\right), 
\\ \DIV(\mu\bH) &= 0, \\
+BC+IC. &
\end{align*}
\section{Analytical solution}
\begin{align*}
H_r(r,\theta,z,t) &= \sum_{m=1}^{3} \frac{1}{m^3}\left(\alpha_m zr^{m-1} m\cos(m\theta) + \beta_m zr^{m-1} m\sin(m\theta)\right)\cos(t), \\
H_{\theta}(r,\theta,z,t) &=  \sum_{m=1}^{3} \frac{1}{m^3}\left(\beta_m zr^{m-1} m\cos(m\theta) - \alpha_m zr^{m-1} m\sin(m\theta)\right)\cos(t),\\
H_z(r,\theta,z,t) &=  \sum_{m=1}^{3} \frac{1}{m^3}\left(\alpha_m r^m \cos(m\theta) + \beta_m r^m \sin(m\theta) \right)\cos(t),\\
\end{align*}
\section{Data file}
\verbatiminput{\data_dir/debug_data_10}
%%%%END OF TEST 10
%
%
%
\testChapter{mhd + temperature (without vacuum).}
%\chapter{Test \testNumber: mhd + temperature (without vacuum).}
\section{Numerical domain and equations to solve}
\section{Analytical solution}
\section{Data file}
\verbatiminput{\data_dir/debug_data_11}
%%%%END OF TEST 11
%
%
%
\testChapter{mxw Dirichlet/Neumann (without vacuum).}
%\chapter{Test \testNumber: mxw Dirichlet/Neumann (without vacuum).}
\section{Numerical domain and equations to solve}
\begin{align*}
\partial_t\left(\mu\bH\right)+\frac{1}{\Rm}\ROT\left(\frac{1}{\sigma}\ROT\bH\right) -
\ROT\left(\bu\CROSS\mu\bH\right) &=
\frac{1}{\Rm}\ROT\left(\frac{1}{\sigma}\bj\right), 
\\ \DIV(\mu\bH) &= 0, \\
+BC+IC. &
\end{align*}
\section{Analytical solution}
\section{Data file}
\verbatiminput{\data_dir/debug_data_12}
%%%%END OF TEST 12
%
%
%
\testChapter{mxw Dirichlet/Neumann + perio (without vacuum).}
%\chapter{Test \testNumber: mxw Dirichlet/Neumann + perio (without vacuum).}
\section{Numerical domain and equations to solve}
\begin{align*}
\partial_t\left(\mu\bH\right)+\frac{1}{\Rm}\ROT\left(\frac{1}{\sigma}\ROT\bH\right) -
\ROT\left(\bu\CROSS\mu\bH\right) &=
\frac{1}{\Rm}\ROT\left(\frac{1}{\sigma}\bj\right), 
\\ \DIV(\mu\bH) &= 0, \\
+BC+IC. &
\end{align*}
\section{Analytical solution}
\section{Data file}
\verbatiminput{\data_dir/debug_data_13}
%%%%END OF TEST 13
%
%
%
\testChapter{mxw + arpack (without vacuum).}
%\chapter{Test \testNumber: mxw + arpack (without vacuum).}
\section{Numerical domain and equations to solve}
\section{Reference results}
\section{Data file}
\verbatiminput{\data_dir/debug_data_14}
%%%%END OF TEST 14
%
%
%
\testChapter{nst (with LES).}
%\chapter{Test \testNumber: nst (with LES).}
\section{Numerical domain and equations to solve}
\section{Reference results}
\section{Data file}
\verbatiminput{\data_dir/debug_data_15}
%%%%END OF TEST 15
%
%
%
\testChapter{??.}
%\chapter{Test \testNumber: ??.}
\section{Numerical domain and equations to solve}
\section{Reference results}
\section{Data file}
\verbatiminput{\data_dir/debug_data_15}
%%%%END OF TEST 16
%
%
%
\testChapter{mxw with vacuum + variable mu}
The purpouse of this case is to test variable permeability
with dependence in $r$ and $z$. There is one conducting region 
embedded in vaccum.
\section{Numerical domain and equations to solve}
The domain is a cylinder $r\in [0,1]$, $z\in [-1,1]$.
The exterior boundary of the vaccum is a sphere of radius 10. Let us recall the kynematic dynamo equations,
\begin{align}
\begin{cases}
\label{eq:maxwell-pde}
\partial_t\left(\mu\bH\right)+\frac{1}{\Rm}\ROT\left(\frac{1}{\sigma}\ROT\bH\right)
- \ROT\left(\bu\CROSS\mu\bH\right) =
\frac{1}{\Rm}\ROT\left(\frac{1}{\sigma}\bj\right), \\ \DIV(\mu\bH) =
0, \\ +BC+IC.  +\phi
\end{cases}
\end{align}

\section{Analytical solution}

Let
\begin{equation}
 \label{Hsol-1}
\mathbf{H}=\frac{1}{{\mu}^c} \nabla \psi,
\end{equation}

\noindent where $\psi=\psi(r,z)$ and satisfies the Laplace equation in cylindrical coordinates,
\begin{equation}
\label{eq:laplace_cyl}
\partial _{rr}\psi + \frac{1}{r} \partial _r \psi + \partial_{zz} \psi = 0.
\end{equation}

\noindent If we also set  $\mathbf{j}= \nabla \times \mathbf{H}$, $\mathbf{u}=0$, $\mathbf{E}=\mathbf{0}$ and $\phi(r,\theta,z,t)=\psi(r,z)$, then $\mathbf H$, defined as in (\ref{Hsol-1}),
satisfies Maxwell equations (\ref{eq:maxwell-pde}).\\



\noindent Now, let
\begin{equation}
{\mu^c}={\mu^c(r,z)}=\frac{1}{f(r,z)+1 },
\end{equation}

\noindent where $$f(r,z)= b \cdot r^3 \cdot (1-r)^3 \cdot (z^2-1)^3,$$
 
\noindent and  $b\geq 0 $ is a  parameter which determines the variation of ${\mu}^c$.
\noindent Observe that 
$$
\partial _r f(r,z) = 3b\big ( r(1-r)\big )^2(1-2r)(z^2-1)^3,
\quad
\partial _z f(r,z) = 6bz\big  (r(1-r))^3 (z^2-1)^2.
$$



Moreover,  $f (r,z) \leq 0$ for $(r,\theta,z) \in \Omega^c$ and,
$$
\sup _{\Omega^c} f(r,z)=f_{\text{max}}=0 , 
\quad 
\inf _{\Omega^c} f(r,z)=f_{\text{min}}= -\frac{b}{2^6}, 
$$

\noindent then,
$$
\mu_{\text{min}} ^c = \frac{1}{1 + f_{\text{max}}},\quad   \mu_{\text{max}} ^c =\frac{1}{1 + f_{\text{min}}},
 \quad  \quad r_{\mu}=\frac{\mu _{\text{max}}}{\mu _{\text{min}}}=\frac{\frac{1}{1-\frac{b}{2^6}}}{1}, \quad \text{and} \quad b= 2^6\left (1- \frac{1}{r_{\mu}}\right).
$$



\noindent To get an explicit solution in (\ref{Hsol-1}),  equation (\ref{eq:laplace_cyl}) is solved using separation of variables, this is, letting
$\psi(r,z)=R(r)Z(z)$ we solve the following system of ODEs,
\begin{eqnarray*}
Z''-\lambda Z & = & 0 \\
R''+\frac{R'}{r}+\lambda  R & = & 0,
\end{eqnarray*}
where $\lambda$ is any real number. Here we choose $\lambda=1$, so
\begin{equation}
\label{psi-sol1}
\psi(r,z)=J_0(r)\text{cosh}(z).
\end{equation}

\noindent Now, using $J_0'(r)=-J_1(r)$ and $\text{cosh}'(z)=\text{sinh}(z)$  we get,
\begin{equation}
\nabla \psi = \left[ \begin{array}{c} -J_1 (r)  \text{cosh}(z) \\ 0 \\J_0 (r)  \text{sinh}(z) \end{array} \right]
\end{equation}

\noindent then by (\ref{Hsol-1}), 

\begin{equation}
\mathbf{H}^c=(f(r,z)+1)
\left[ \begin{array}{c} -J_1 (r)  \text{cosh}(z) \\ 0 \\J_0 (r)  \text{sinh}(z) \end{array} \right],
\end{equation}

\noindent To get $\nabla \times \mathbf{H}$, we use the identity 
$$
\nabla \times \left (\frac{1}{{\mu}^c} \nabla \psi \right )= \nabla \left ( \frac{1}{{\mu}^c} \right ) \times  \nabla
\psi + \frac{1}{{\mu}^c} \nabla \times  \nabla \psi,
$$
but $\nabla \times  \nabla \psi = 0$. Then  using  equation (\ref{Hsol-1}), 

\begin{equation*}
\nabla \times \mathbf{H}^c=\nabla \left ( \frac{1}{{\mu}^c} \right ) \times  \nabla \psi, 
\end{equation*}

\noindent and 
\begin{equation}
 \nabla  \frac{1}{{\mu}^c}  =\left[ \begin{array}{c} \partial _r f(r,z) \\ 0  \\ \partial _z f(r,z) \end{array} \right];
\end{equation}

\noindent we obtain,

\begin{equation}
\mathbf{j} = \nabla \times \mathbf{H}^c=
\left[ \begin{array}{c}
0 \\ 
-\partial _r f(r,z) J_0(r)   \text{sinh}(z)  
-\partial _z f(r,z) J_1(r)    \text{cosh}(z) \\
 0
\end{array} \right].
\end{equation}

\noindent In summary,


\begin{equation}
\label{psi-sol1}
\phi(r,\theta,z,t)=J_0(r)\text{cosh}(z).
\end{equation}

\begin{equation}
\mathbf{H}^c= \left ( 2^6\left (1- \frac{1}{r_{\mu}}\right)\cdot r^3 \cdot (1-r)^3 \cdot (z^2-1)^3 +1\right)
\left[ \begin{array}{c} -J_1 (r)  \text{cosh}(z) \\ 0 \\J_0 (r)  \text{sinh}(z) \end{array} \right],
\end{equation}


\section{Reference results}
\section{Data file}
\verbatiminput{\data_dir/debug_data_17}
%%%%END OF TEST 15
%
%
%


\end{document}
